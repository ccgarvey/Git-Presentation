\documentclass[12pt]{article}
\usepackage{hyperref}

\begin{document}hyperref
Setting username and email:
git config [flags] user.name "Git User"
git config [flags] user.email gitisgreat@domain.com
-You can use the flag --global if you want to set these for the whole system

Start a new repository:
git init
-It doesn't matter if you have files in the directory or not
-This creates a .git folder, which will store metadata about your files

Check the state of your repository:
git status
-If you don't want to see about an untracked file, you can make a file called
    .gitignore, which will not remind you about the untracked files you list
    in it
-Contains lots of helpful messages

Stage a change for committing:
git add [flag(s)] filename
-* is a wildcard in the file name
-The -N argument allows you show your intent to add them, which allows you to
    later use -a for new files.

Unstage a file:
git reset HEAD filename
-This does not change the file in your working directory, but reset can with
    different arguments, so be careful using it

Commit file(s):
git commit [flags]
-The flag -a automatically stages modified/deleted files for this commit
-The flag -m allows you to write a commit message in-line:
    git commit -m "Message here"
-The flag --amend can be used if you goof up a commit. See
    http://git-scm.com/book/en/Git-Basics-Undoing-Things -- this can mess
    things up if you do it wrong.

See the changes to a file:
git diff [commit hash] [filename]
-This is the diff to the last staged (or committed if none are staged) version
    (this means that if you stage then run diff, it won't show any difference)
-The flags --staged or --cached can be used to show difference between the
    currently staged files and previous commit(s)

Stage file removal and remove file from directory:
git rm [flags] filename
-directly deleting from the actual directory makes an unstaged change
-The flag --cached removes the staged snapshot of a file, but not the file
    itself

See commit history:
git log [flags]
-Commits are displayed in reverse chronological order
-The flag -<number n> shows the last n commits (eg. git log -5)
-The flag -p shows the diffs for the commits
-There are LOTS of log options, check out the book section on it!
    http://git-scm.com/book/en/Git-Basics-Viewing-the-Commit-History
-The --oneline flag shortens output to one line per commit
-The --graph flag shows a neat graphical representation of the history

Update files in the current directory:
git checkout <branch or (-- file{\_}path)> [commit]
-Used to change the current branch to another branch or change a file to a
    previous version
-This discards the changes in the current working directory, so be careful
    with it. You may wish to look up how to use it while gaining familiarity
-The -- is required for files, because file and branch names can be the same
    
Create a local copy of an existing repository:
git clone repository{\_}url
-This repo's URL is \url{https://github.com/ccgarvey/Git-Presentation}

See a list of the remote repositories:
git remote [flags]
-The flag -v shows the URL for each remote
-The flag/phrase "-remote add shortname URL" adds a repository with an
    abbreviated name
-The flag/phrase "-remote rename" renames a short name)

Get data from a remote repository that you don't already have:
git fetch remote{\_}name
-This puts the data in the .git folder, so it doesn't modify your own files
-You need to merge later if there are any conflicts (or if you want branches
    you don't have yet)
    
Fetch and merge a repository at the same time:
git pull remote{\_}name
-in older versions of git, you should commit before pulling (so do it always)

Share your work on the project:
git push
-If you clone, fetch, or pull and someone else pushes first, you need to pull
    before you can push your commits (it doesn't allow you to discard things)

Create, delete, or get information about branches:
git branch [flags] [branch name]
-With no arguments, git branch lists the current branches
-With just a branch name (ex. "experimental"), creates
-To switch between branches, use checkout
    -You must commit changes before switching
-The flag -d deletes a branch
-Branches must be pushed to share:
    git push remote{\_}name localbranch[:remotebranch]

Merge disparate files or branches:
git merge
-If you get an auto-merge with conflicts, you get info in your files you
    must resolve, you can use git mergetool to merge if you would like
-You must commit your merge after completing it.
-Don't merge with uncommitted changes in your local directory!
-When you fetch and it gets remote branches, you must merge the branches in
-rebasing is another way to combine divergent histories (see below)

Merge branch in a way that produces linear commit history (rebase):
git rebase branch{\_}to{\_}rebase{\_}on
-NEVER EVER EVER PUSH REBASED CODE TO A PUBLIC REPOSITORY.
    -"If you [do], people will hate you, and you’ll be scorned by friends and
     family." --Git Pro, \url{http://git-scm.com/book/en/Git-Branching-Rebasing}
-This rebases your current branch onto the branch specified
-This finds the common ancestor of the two branches involved, finds the
    differences between the two, and applies them to the branch being rebased
    onto so that there are new commits ending with the files as they are in your
    current branch (if this is confusing, try the git pro book link provided)
-When the rebase is done, merge is used and automatically combines the branches:
    git checkout branch{\_}rebased{\_}on
    git merge
-This produces a linear history, instead of a branched-joined history
    
Get help about commands:
git help [command]
git [command] --help
man git [command]


Websites for hosting git code:
-GitHub: \url{https://github.com/}
-BitBucket: \url{https://bitbucket.org/}

\end{document}